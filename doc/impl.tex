\documentclass{article}

\usepackage{amsmath}
\usepackage{bbm}

\newcommand{\setR}{\mathbbm{R}}

\begin{document}
Let us assume that $E$ is a $d$-dimensional polytope and its boundary $\partial E$
is partitioned into $(d-1)$-dimensional polytopes $e$ with constant normal $\nu_e$.
By abuse of Notation, we denote these polytopes by $e \in \partial E$.

We first compute the $L^2$-Projection $\Pi_p \Phi_i \in \mathcal{P}_p( E )$
\begin{equation*}
  \int_E \Pi_p \Phi_i\,\varphi = \int_E \Phi\,\varphi
  \qquad\text{for all $\varphi \in \mathcal{P}_p( E )$.}
\end{equation*}

Using a basis $\varphi_1, \ldots, \varphi_{N_p}$ of $\mathcal{P}_p( E )$ and using
the representation
\begin{equation*}
  \Pi_p \Phi_i = \sum_{\alpha=1}^{N_p} \Pi^0_{i,\alpha}\,\varphi_\alpha
\end{equation*}
we obtain
\begin{equation*}
  \sum_{\alpha=1}^{N_p} \Pi^0_{i, \alpha}\,\int_E \varphi_\alpha\,\varphi_\beta
  = \int_E \Pi_p \Phi_i\,\varphi
  = \int_E \Phi\,\varphi_i
  \qquad\text{for $\beta = 1, \ldots, N_p$.}
\end{equation*}
The matrix $(\Pi^0_{i, \alpha})_{i, \alpha}$ is stored as \texttt{valueProjection}
in the VEM basis function set.


Now we turn to the computation of $\Pi_{p-1} \nabla \Phi_i$.
For $\alpha = 1, \ldots, N_{p-1}$, we have
\begin{equation*}
  \int_E \Pi_{p-1} \nabla \Phi_i\,\varphi_\alpha
    = \int_E \nabla \Phi_i\,\varphi_\alpha
    = \int_{\partial E} \Phi_i\,\varphi_\alpha\,\nu - \int_E \Phi_i\,\nabla \varphi_\alpha.
\end{equation*}
Since $\nabla \varphi_\alpha \in \mathcal{P}_p( E )^d$, the following identity holds:
\begin{equation*}
  \int_E \Phi_i\,\nabla \varphi_\alpha
    = \int_E \Pi_p \Phi_i\,\nabla \varphi_\alpha
    = \sum_{\beta=1}^{N_p} \Pi^0_{i, \beta}\,\int_E \varphi_\beta\,\nabla \varphi_\alpha.
\end{equation*}
For the boundary integral, we use the fact that $\varphi_\alpha \in \mathcal{P}_p( e )$
for all $e \in \partial E$.
We therefore obtain
\begin{equation*}
  \int_{\partial E} \Phi_i\,\varphi_\alpha\,\nu
    = \sum_{e \in \partial E} \nu_e\,\int_e \Phi_i\,\varphi_\alpha
    = \sum_{e \in \partial E} \nu_e\,\int_e \Pi_p\,\Phi_i\,\varphi_\alpha,
\end{equation*}
i.e., we use the $L^2$-projection in the VEM-Space $V_e$.
Notice that by definition $\Phi_i$ either codincides on $e$ with a basis function
$\Psi_j$ of $V_e$ or vanishes completely.
As before, we represent $\Pi_{p-1}\,\nabla \Phi$ in the basis
$\varphi_1, \ldots, \varphi_{N_{p-1}}$ with coefficients in
$\Pi^1_{i,\alpha} \in \setR^d$
\begin{equation*}
  \Pi_{p-1} \nabla \Phi_i = \sum_{\alpha=1}^{N_{p-1}} \Pi^1_{i,\alpha}\,\varphi_\alpha
\end{equation*}
to finally obtain for $\beta = 1, \ldots, N_{p-1}$:
\begin{equation*}
  \sum_{\beta=1}^{N_{p-1}} \Pi^1_{i, \beta}\,\int_E \varphi_\beta\,\varphi_\alpha
  = \sum_{e \in \partial E} \nu_e\,\int_e \Pi_p\,\Phi_i\,\varphi_\alpha
    - \sum_{\beta=1}^{N_p} \Pi^0_{i, \beta}\,\int_E \varphi_\beta\,\nabla \varphi_\alpha
\end{equation*}


\end{document}
